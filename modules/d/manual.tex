\documentclass[a4paper]{report}

% begin config
\author{hephaisto}
\newcommand{\moduleTitle}{Blink}
\newcommand{\moduleRevision}{0}
\newcommand{\moduleId}{d}
% end config

\title{\moduleTitle}
\usepackage[utf8]{inputenc}
\usepackage{graphicx}
\usepackage[a4paper]{geometry}

\usepackage{fancyhdr}
\pagestyle{fancy}
\lhead{BUMM defusal manual}
\rhead{ {\moduleTitle} module (Rev. \moduleRevision)}
\cfoot{}

\renewcommand{\thefigure}{\moduleId.\arabic{figure}} 
\renewcommand{\thetable}{\moduleId.\arabic{table}} 

\renewcommand{\familydefault}{\ttdefault}

\newcommand{\warning}[1]{\paragraph{WARNING} #1}

\usepackage{tikz}

\begin{document}

% begin content

\begin{figure}[t]
	\begin{center}
		\begin{tikzpicture}[scale=0.03,y=-1cm]
			\input{layout_autogen}
		\end{tikzpicture}
	\end{center}
	\caption{Panel layout}
	\label{fig:d_layout}
\end{figure}

\en{
The blink module shows the user a message by blinking a certain code with a red and a green light.
All possible message codes are listed in the code book below, which matches each message code to a command.
In the code book, each entry contains $N+1$ letters, the first $N$ being either r or g depending on the blink code shown (left character first).
The last digit gives the command the code translates to.
Depending on the command, different actions have to be taken to disable the module according to table \ref{tab:d_commands}.
For example, the entry rgR means: If you see a single red blink followed by a single green blink, press the red button.
\warning{Often multiple iterations of the above actions are necessary!}
}
\de{
Das ``blink'' Modul zeigt dem Benutzer einen Code, indem rote und grüne Lampen in einem bestimmten Rhythmus aufleuchten.
Alle möglichen Codes sind im Codebuch im Anhang aufgelistet, das jedem Code ein Kommando zuordnet.
Jeder Eintrag im Codebuch besteht aus $N+1$ Zeichen, wobei die ersten $N$ Zeichen entweder r oder g lauten, und für eine entsprechend gefärbte Lampe stehen. Hierbei wird von links nach rechts gelesen.
Das letzte Zeichen im Eintrag gibt an, welches Kommando ausgeführt werden muss.
Die genauen Handlungen für jedes Kommando sind in Tabelle \ref{tab:d_commands} aufgeführt.

Beispiel: Der Eintrag rgR bedeutet: Wenn zuerst die rote Lampe aufblinkt und dann die grüne, dann muss der rote Knopf gedrückt werden.
\warning{Meist muss das obige Vorgehen mehrmals mit verschiedenen Codes wiederholt werden!}
}
\chde{
Die ``Blink'' Modul zeigt den Benutzer durch die rote und grüne Lichter in der Code ein bestimmter Rhythmus ist.
Alle möglichen Codes werden in der Anlage zu dem Codebuch enthalten sind, wobei jeder Code auf den Befehl zugeordnet.
In jeder Eingabe des Codes durch die $N+1$ Marke, mit den ersten $N$ Zeichen sind die Buchstaben r oder g laut, und die entsprechende Farbe des Lichts. Dies ist der richtige von links zu lesen.
Der letzte Eintrag in einem Zeichen angegebenen Befehl verarbeitet werden müssen.
Die genaue Aktionstabelle für jeden Befehl \ref{tab:d_commands} Beschreibung.

Beispiel: Geben Sie rgR bedeutet: Erstens, blinkt rot, dann grün, dann rot-Taste gedrückt werden muss.
\warning{unterschiedlichsten Codes müssen Sie diesen Vorgang mehrmals wiederholen! }
}


\begin{table}
	\centering
	\begin{tabular}{|cp{5cm}|}\hline
		\input{manual_description_autogen}\hline
	\end{tabular}
	\caption{\en{Message commands}\de{Kommandos}}
	\label{tab:d_commands}
\end{table}

\input{manual_tables_autogen}

% end content

\end{document}
